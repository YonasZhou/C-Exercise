\nsection{The Game of Life}

\input{tikzsets}

The Game of Life was developed by British mathematician
John Horton Conway. In Life, a board represents the world
and each cell a single location. A cell may be either empty
or inhabited. The game has three simple rules, which relate to the
cell's eight nearest neighbours~:
\begin{enumerate}
\item {\bf Survival} An inhabited cell remains inhabited if
exactly $2$ or $3$ of its neighbouring cells are inhabited.
\item {\bf Death} An inhabited cell becomes uninhabited if 
fewer than $2$, or more than $3$ of its neighbours are inhabited.
\item {\bf Birth} An uninhabited cell becomes inhabited if exactly
$3$ of its neighbours are inhabited.
\end{enumerate}

The next board is derived solely from the current one. The current board
remains unchanged while computing the next board.  In the simple case
shown here, the boards alternate infinitely between these two states.

\begin{center}
\begin{tikzpicture}
\matrix [noughtsone board]
{
 & & & &  \\
 & &1& &  \\
 & &1& &  \\
 & &1& &  \\
 & & & &  \\
};
\end{tikzpicture}
\begin{tikzpicture}
\matrix [noughtsone board]
{
 & & & &  \\
 & & & &  \\
 &1&1&1&  \\
 & & & &  \\
 & & & &  \\
};
\end{tikzpicture}
\end{center}

\subsection*{The 1.06 format}

A general purpose way of encoding the input board is
called the Life 1.06 format~:
\wwwurl{http://conwaylife.com/wiki/Life_1.06}
This format has comments indicated by a hash in the first column,
and the first line is always:
\begin{terminaloutput}
#Life 1.06
\end{terminaloutput}
Every line specifies an $x$ and $y$ coordinate of a live cell;
such files can be quite long.
The coordinates specified are relative to the middle of the board,
so~:
\begin{verbatim}
0  -1
\end{verbatim}
means the middle row, one cell to the left of the centre.

There are hundreds of interesting patterns stored like this
on the above site.

\begin{exercise}
\label{ex:life106}
Write a program which is run using the \verb^argc^ and
\verb^argv^ parameters to \verb^main^. The usage is
as follows~:
\begin{terminaloutput}
% life file1.lif 10
\end{terminaloutput}
where \verb^file1.lif^ is a file specifying the inital
state of the board, and \verb^10^ specifys that ten
iterations are required.

Display the output to screen every iteration using plain text,
you may assume that the board is $150$ cells wide and $90$ cells tall.
\end{exercise}

\subsection*{Alternative Rules for The Game of Life}

The rules for life could also be phrased in a different manner, that
is, give birth if there are two neighbours around an empty cell (B2)
and allow an `alive' cell to survive only if surrounded by 2 or 3 cells (S23).
Other rules which are {\it life-like} exist,
for instance {\it 34 Life} (B34/S34), {\it Life Without Death} (B3/S012345678)
and {HighLife} (B36/S23).
\wwwurl{http://en.wikipedia.org/wiki/Life-like_cellular_automaton}

\begin{exercise}
Write a program that allows the user to input life-like rules e.g.~:
\begin{terminaloutput}
life B34/S34 lifeboard.lif
\end{terminaloutput}
or
\begin{terminaloutput}
life B2/S lifeboard.lif
\end{terminaloutput}
and display generations of boards, beginning with the inital board in
the input file.
\end{exercise}
