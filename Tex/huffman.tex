\nsection{Huffman Encoding}
Huffman encoding is commonly used for data compression.
Based on the frequency of occurrence of characters, you build
a tree where rare characters appear at the bottom of the tree, and
commonly occurring characters are near the top of the tree.

For an example input text file, a Huffman tree might look something like:

{\small
\begin{verbatim}
010 :        00101 (   5 *  125)
` ' :          110 (   3 *  792)
`"' :    111001010 (   9 *   12)
`'' :     00100000 (   8 *   15)
`(' :  01100000100 (  11 *    2)
`)' :  01100001101 (  11 *    2)
`,' :      1001001 (   7 *   39)
`-' :      0010010 (   7 *   31)
`.' :      1001100 (   7 *   40)
`/' :  00100110000 (  11 *    1)
`0' :  11100110010 (  11 *    3)
`1' :     00100010 (   8 *   15)
`3' :  01100000101 (  11 *    2)
`4' :  01100001001 (  11 *    2)
`5' :  11100110011 (  11 *    3)
`6' :  01100001000 (  11 *    2)
`7' :  01100001100 (  11 *    2)
`8' :    001001101 (   9 *    8)
`9' :     10010000 (   8 *   18)
`:' :  01100001011 (  11 *    2)
`A' :     00100111 (   8 *   16)
`B' :    111001101 (   9 *   13)
`C' :     10011011 (   8 *   22)
`D' :    111001110 (   9 *   13)
`E' :     10011010 (   8 *   19)
`F' :    111001000 (   9 *   11)
`G' :   0110000000 (  10 *    4)
`H' :   1110011111 (  10 *    7)
`I' :   1110010011 (  10 *    6)
`J' :  11100111101 (  11 *    3)
`K' :   1110010111 (  10 *    6)
`L' :     00100011 (   8 *   15)
`M' :  11100111100 (  11 *    3)
`N' :  01100001010 (  11 *    2)
`O' :  01100000111 (  11 *    2)
`P' :   1110011000 (  10 *    6)
`R' :   0110000111 (  10 *    5)
`S' :     10010001 (   8 *   19)
`T' :   0010011001 (  10 *    4)
`U' :   1110010010 (  10 *    5)
`W' :   0110000001 (  10 *    4)
`a' :         1010 (   4 *  339)
`b' :      1111110 (   7 *   60)
`c' :       100101 (   6 *   77)
`d' :        01101 (   5 *  143)
`e' :          000 (   3 *  473)
`f' :       100111 (   6 *   84)
`g' :       111000 (   6 *   94)
`h' :        11110 (   5 *  223)
`i' :         0100 (   4 *  266)
`j' :  01100000110 (  11 *    2)
`k' :     00100001 (   8 *   15)
`l' :        10110 (   5 *  176)
`m' :       101111 (   6 *   92)
`n' :         0111 (   4 *  288)
`o' :         0101 (   4 *  269)
`p' :       101110 (   6 *   89)
`q' :  00100110001 (  11 *    2)
`r' :        11101 (   5 *  214)
`s' :         0011 (   4 *  260)
`t' :         1000 (   4 *  305)
`u' :       111110 (   6 *  108)
`v' :      0110001 (   7 *   37)
`w' :      1111111 (   7 *   60)
`x' :   1110010110 (  10 *    6)
`y' :       011001 (   6 *   72)
2916 bytes
\end{verbatim}
}

Each character is shown, along with its Huffman bit-pattern, the length
of the bit-pattern and the frequency of occurrence. At the bottom, the total
number of bytes required to compress the file is displayed.

\begin{exercise}
Write a program that reads in a file (argv[1]) and, based on the characters
it contains, computes the Huffman tree, displaying it as above. 
\end{exercise}
