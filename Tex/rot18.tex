\nsection{Rot18}



ROT18 is an encryption algorithm made of the combination of ROT13 and
ROT5. ROT13 replaces each letter in a string with the letter 13 places
later in the alphabet, wrapping back to the beginning of the alphabet
where necessary e.g. 'a' becomes 'n', 'b' becomes 'o', and 'n' becomes
'a'. ROT5 replaces each digit in a string with the digit 5 greater,
wrapping back to 0 where necessary e.g. '0' becomes '5', '1' becomes
'6', and '5' becomes '0'.

\wwwurl{https://www.boxentriq.com/code-breaking/rot13}

\begin{exercise}

Write the function whose "top-line" is:
\begin{codesnippet}
void rot(char str[])
\end{codesnippet}

\noindent that takes a string as input and encrypts it using the ROT18
cipher. The function must preserve the case of any letters, and any
characters that aren't letters or digits must be unaffected.

When the string \verb^"Hello, World!"^ is encrypted, it becomes
\verb^"Uryyb, Jbeyq!"^ and when encrypted a second time, reverts to the
original string.


\begin{verbatim}
Erzrzore:  fubeg shapgvbaf; ubhfr-fglyr ehyrf :-)
\end{verbatim}

\end{exercise}

